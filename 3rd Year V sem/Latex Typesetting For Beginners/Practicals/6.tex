\documentclass[12pt]{article}

\title{An Overview of Key Computer Science Books}
\author{Sudeep Kumar Singh}
\date{\today}

\begin{document}

\maketitle

\section{Introduction}
This document highlights five foundational books in computer science and software development. These books cover topics from programming and algorithms to software engineering principles.

One of the most highly regarded texts is "The Art of Computer Programming" by Donald Knuth \cite{knuth1997art}. Additionally, Robert C. Martin's "Clean Code" \cite{martin2008clean} is essential for software developers aiming to write maintainable code.

\section{Overview of Important Books}
Computer science students frequently consult "Introduction to Algorithms" by Cormen et al. for algorithms \cite{cormen2009algorithms}. Furthermore, for an understanding of operating systems, "Operating System Concepts" by Silberschatz et al. \cite{silberschatz2018os} provides extensive knowledge. 

For those interested in software design patterns, "Design Patterns: Elements of Reusable Object-Oriented Software" by Gamma et al. \cite{gamma1994design} is a classic text that offers practical solutions.

\section{Conclusion}
These books collectively provide a well-rounded foundation in computer science, making them valuable resources for both students and professionals.

\begin{thebibliography}{99}

\bibitem{knuth1997art}
D. E. Knuth, \textit{The Art of Computer Programming}. Addison-Wesley, 1997.

\bibitem{martin2008clean}
R. C. Martin, \textit{Clean Code: A Handbook of Agile Software Craftsmanship}. Prentice Hall, 2008.

\bibitem{cormen2009algorithms}
T. H. Cormen, C. E. Leiserson, R. L. Rivest, and C. Stein, \textit{Introduction to Algorithms}. MIT Press, 2009.

\bibitem{silberschatz2018os}
A. Silberschatz, P. B. Galvin, and G. Gagne, \textit{Operating System Concepts}. Wiley, 2018.

\bibitem{gamma1994design}
E. Gamma, R. Helm, R. Johnson, and J. Vlissides, \textit{Design Patterns: Elements of Reusable Object-Oriented Software}. Addison-Wesley, 1994.

\end{thebibliography}

\end{document}
