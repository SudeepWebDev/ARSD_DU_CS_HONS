\documentclass[12pt]{report}

% Packages for customization
\usepackage{tocloft}     % Customizing the table of contents, list of figures, and list of tables
\usepackage{makeidx}     % Creating an index
\usepackage{graphicx}    % Including images (for figures)

% Index initialization
\makeindex

% Customizing the table of contents and lists
\renewcommand{\cftsecfont}{\bfseries}        % Bold section titles in TOC
\renewcommand{\cftsecpagefont}{\bfseries}    % Bold page numbers in TOC
\renewcommand{\cftfigfont}{\itshape}         % Italicize figure entries in list
\renewcommand{\cfttabfont}{\itshape}         % Italicize table entries in list
\renewcommand{\cftfigpagefont}{\itshape}     % Italicize page numbers in figure list
\renewcommand{\cfttabpagefont}{\itshape}     % Italicize page numbers in table list

\title{Advanced Document with TOC, Index, and Lists}
\author{Sudeep Kumar Singh}
\date{\today}

\begin{document}

\maketitle

\tableofcontents % Generate Table of Contents
\listoffigures   % Generate List of Figures
\listoftables    % Generate List of Tables

\chapter{Introduction}
LaTeX is a robust typesetting system ideal for producing structured documents. This document demonstrates customizations for the Table of Contents (TOC), lists of figures and tables, and generating an index.

% Adding keywords to index
LaTeX\index{LaTeX} is widely used in academia\index{academia}. It supports a range of features for document preparation, including index creation\index{index} and customization of TOC\index{Table of Contents} entries.

\chapter{Customizing Lists}
The `tocloft` package allows modifying the appearance of TOC, list of figures\index{list of figures}, and list of tables\index{list of tables} entries. We can adjust font styles and spacing for a more polished look.

\section{Adding Figures and Tables}
Below is a sample figure and table to illustrate list customizations.

\begin{figure}[h!]
    \centering
    \includegraphics[width=0.5\textwidth]{example-image}
    \caption{Sample Figure}
    \index{Sample Figure}
\end{figure}

\begin{table}[h!]
    \centering
    \caption{Sample Table}
    \begin{tabular}{|c|c|}
        \hline
        \textbf{Item} & \textbf{Description} \\
        \hline
        1 & Example entry \\
        2 & Another example entry \\
        \hline
    \end{tabular}
    \index{Sample Table}
\end{table}

\chapter{Creating an Index}
An index helps readers find keywords and relevant sections quickly. Using the `makeidx` package, you can add entries by inserting `\texttt{\textbackslash index\{keyword\}}` commands throughout the document.

\chapter{Conclusion}
This document illustrates how to use LaTeX for customizing TOC, lists, and index creation. By applying the `tocloft` and `makeidx` packages, documents can be tailored for both style and functionality, ensuring they meet specific academic or professional standards.

\printindex % Generate the Index

\end{document}
