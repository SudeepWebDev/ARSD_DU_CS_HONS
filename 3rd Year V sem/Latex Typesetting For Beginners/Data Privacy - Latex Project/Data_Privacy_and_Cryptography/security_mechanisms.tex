\thispagestyle{empty}

\begin{center}
    \vspace*{\fill}
    \Huge \textbf{Chapter 3} \\
    \Huge \textbf{Security Mechanisms}
    \vspace*{\fill}
\end{center}

\newpage
\chapter{Security Mechanisms}

\section{Key Security Mechanisms}
Security mechanisms are the methods used to implement security services and ensure the protection of information systems. The key security mechanisms are as follows:

\subsection{Cryptographic Algorithms}
These are techniques used to transform data into a form that is unintelligible to unauthorized parties. Cryptographic algorithms are covered in detail in chapter 4.

\subsection{Data Integrity}
Mechanisms ensuring the integrity of a data unit or stream of data units. These mechanisms verify that data has not been altered during transmission or storage.

\subsection{Digital Signature}
A digital signature is data appended to or a cryptographic transformation of a data unit, allowing the recipient to prove the source and integrity of the data unit. It also protects against forgery by ensuring authenticity.

\subsection{Authentication Exchange}
This mechanism ensures the identity of an entity through the exchange of information, allowing parties to confirm each other’s identity.

\subsection{Traffic Padding}
Traffic padding involves the insertion of bits into gaps in a data stream to obscure the size or frequency of the data, thus frustrating traffic analysis attempts. This makes it harder for attackers to discern patterns in the communication.

\subsection{Routing Control}
Routing control enables the selection of secure routes for data and allows routing changes, especially when a breach of security is suspected. It helps in directing the flow of sensitive information through the most secure channels.

\subsection{Notarization}
Notarization involves a trusted third party that verifies certain properties of a data exchange, ensuring that data has not been tampered with and confirming its authenticity.

\subsection{Access Control}
Access control mechanisms enforce access rights to resources, ensuring that only authorized users or systems can access or modify certain information or resources.

