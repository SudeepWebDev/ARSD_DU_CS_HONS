% Set up the page style

\fancyfoot[LO,RE]{\small
    \textbf{NIST\textsuperscript{1}:} National Institute of Standards and Technology, 
    \textbf{FIPS\textsuperscript{2}:} Federal Information Processing Standards, 
    \textbf{SP\textsuperscript{3}:} Special Publication, 
    \textbf{ENISA\textsuperscript{4}:} European Union Agency for Cybersecurity, 
    \textbf{SHA\textsuperscript{5}:} Secure Hash Algorithm, 
    \textbf{RSA\textsuperscript{6}:} Rivest-Shamir-Adleman,  
    \textbf{DSA\textsuperscript{7}:} Digital Signature Algorithm,  
    \textbf{ECDSA\textsuperscript{8}:} Elliptic Curve Digital Signature Algorithm
}


% Chapter Title Page
\clearpage
\thispagestyle{empty} 
\begin{center}
    \vspace*{\fill} 
    \Huge \textbf{Chapter 8} \\
    \Huge \textbf{Practical Considerations} 
    \vspace*{\fill}
\end{center}
\clearpage

\chapter{Practical Considerations}

\section{Cryptographic Algorithms and Key Lengths}

\subsubsection{Security Over Time:}
\begin{itemize}
    \item Processor speeds and scrutiny increase over time, which can lead old algorithms to become insecure.
    \item Key lengths and hash values that were secure before may now be too weak due to increased computational power and analysis techniques.
\end{itemize}

\subsection{Guidance Sources:}
\begin{itemize}
    \item \textbf{FIPS\textsuperscript{2} 140-2A:} Approved Security Functions for FIPS PUB 140-2.
    \item \textbf{SP\textsuperscript{3} 800-131A:} Transitioning the Use of Cryptographic Algorithms and Key Lengths.
    \item \textbf{ENISA\textsuperscript{4}:} Algorithms, Key Size, and Protocol Report [ECRY18].
\end{itemize}

\subsection{Recommendations:}
\begin{enumerate}
    \item \textbf{Symmetric Encryption:}
    \begin{itemize}
        \item \textbf{Advanced Encryption Standard (AES):} Key lengths of 128, 192, or 256 bits.
    \end{itemize}
    \item \textbf{Hash Functions:}
    \begin{itemize}
        \item \textbf{SHA\textsuperscript{5}-2 or SHA-3:} Hash lengths range from 224 to 512 bits.
    \end{itemize}
    \item \textbf{Digital Signatures:}
    \begin{itemize}
        \item \textbf{Digital Signature Algorithm (DSA)\textsuperscript{7}:} 2048 bits.
        \item \textbf{RSA\textsuperscript{6} Algorithm:} 2048 bits.
        \item \textbf{Elliptic-Curve Digital Signature Algorithm (ECDSA)\textsuperscript{8}:} 224 bits.
    \end{itemize}
\end{enumerate}

\subsection{Implementation Considerations:}
\begin{itemize}
    \item \textbf{Standards Selection:}
    \begin{itemize}
        \item Rely on standardized algorithms (e.g., AES, SHA, DSS).
        \item Use standards developed by \textbf{NIST\textsuperscript{1}} and other trusted organizations.
    \end{itemize}
    \item \textbf{Implementation Methods:}
    \begin{itemize}
        \item Choose between hardware, software, and firmware based on security, cost, simplicity, efficiency, and ease of implementation.
    \end{itemize}
    \item \textbf{Key Management:}
    \begin{itemize}
        \item Administer and manage cryptographic keys (generation, protection, storage, etc.).
        \item For more details, refer to \textit{Effective Cybersecurity: Best Practices and Standards [STAL19]}.
    \end{itemize}
    \item \textbf{Cryptographic Module Security:}
    \begin{itemize}
        \item Secure design, implementation, and use of cryptographic modules.
        \item Use \textbf{NIST\textsuperscript{1}} Cryptographic Module Validation Program (CMVP) for validation.
    \end{itemize}
\end{itemize}

\section{Lightweight Cryptographic Algorithms}

\subsection{Focus:}
\begin{itemize}
    \item Develop secure algorithms minimizing execution time, memory usage, and power consumption.
    \item Suitable for embedded systems (e.g., IoT devices).
\end{itemize}

\subsection{NIST\textsuperscript{1} Project:}
\begin{itemize}
    \item \textbf{NIST\textsuperscript{1}} is working on developing a portfolio of lightweight algorithms.
    \item Initial focus is on symmetric encryption and secure hash functions.
\end{itemize}

\section{Post-Quantum Cryptographic Algorithms}

\subsubsection{Concerns:}
\begin{itemize}
    \item Quantum computers may break current asymmetric cryptographic algorithms, leading to vulnerabilities.
\end{itemize}

\subsection{NIST\textsuperscript{1} Effort:}
\begin{itemize}
    \item \textbf{NIST\textsuperscript{1}} is working on standardizing algorithms that can replace or complement existing asymmetric cryptographic schemes.
    \item They are exploring mathematical approaches for new asymmetric cryptographic algorithms resistant to quantum computing threats.
\end{itemize}


