% Chapter Title Page
\clearpage
\thispagestyle{empty} 
\begin{center}
    \vspace*{\fill} 
    \Huge \textbf{Chapter 4} \\
    \Huge \textbf{Cryptographic Algorithms} 
    \vspace*{\fill}
\end{center}
\clearpage

\chapter{Cryptographic Algorithms}

\section{Definitions}

\textbf{Cryptography:} The discipline that involves the principles, means, and methods for transforming data to hide its content, prevent unauthorized use, or undetected modification. It provides information security, including confidentiality, data integrity, non-repudiation, and authenticity.

\textbf{Cryptographic Algorithm:} A well-defined computational procedure related to cryptography that takes variable inputs, often including a cryptographic key, and produces an output.

\section{Categories of Cryptographic Algorithms}

\begin{table}[h]
\centering
\resizebox{\textwidth}{!}{
\begin{tabular}{|p{3.5cm}|p{9cm}|}
\hline
\textbf{Category} & \textbf{Description} \\
\hline
Keyless & Algorithms that do not use any keys during cryptographic transformations. \\

\hline
Single-Key & Algorithms where the transformation result is a function of the input data and a single secret key. \\
\hline
Two-Key & Algorithms that use two related keys (private key and public key) during different stages of the calculation. \\
\hline
\end{tabular}%
}
\caption{Categories of Cryptographic Algorithms}
\end{table}

\section{Keyless Algorithms}

\subsection{Cryptographic Hash Function:} 
Converts a variable amount of text into a small, fixed-length value called a hash value, hash code, or digest. 
\\
It has properties that make it useful in other cryptographic algorithms, such as message authentication codes or digital signatures.

\subsection{Pseudorandom Number Generator:} 
Produces a deterministic sequence of numbers or bits that appear to be random, sufficient for some cryptographic purposes.

\section{Single-Key Algorithms (Symmetric Encryption)}

\subsection{Symmetric Encryption Algorithms:} 
Use a secret key shared between two parties to encrypt and decrypt data. \\This ensures that communication between the parties is protected from outsiders.

\subsection{Message Authentication Code (MAC):} 
A data element generated by a cryptographic transformation involving a secret key and typically a cryptographic hash function of the message. \\It is used to verify the integrity of the message by those possessing the secret key.

\section{Two-Key Algorithms (Asymmetric Encryption)}

\subsection{Asymmetric Encryption Algorithms:} 
Use two related keys: a private key (known only to the user) and a public key (available to others). It can work in two ways:
\begin{itemize}
    \item Encrypting data with the private key and decrypting with the public key.
    \item Encrypting data with the public key and decrypting with the private key.
\end{itemize}

\section{Applications}

\subsection{Digital Signature Algorithm:} 
A value computed with a cryptographic algorithm associated with a data object, used to verify the data's origin and integrity. The signer uses their private key to generate the signature, and anyone with the public key can verify it.

\subsection{Key Exchange:} 
Securely distributing a symmetric key to two or more parties.

\subsection{User Authentication:} 
Authenticating a user attempting to access an application or service and verifying that the service is genuine.

These cryptographic algorithms are fundamental in ensuring the secure storage, transmission, and interaction of data.

