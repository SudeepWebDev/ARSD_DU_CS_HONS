% Appendix
\clearpage
\thispagestyle{empty} 
\begin{center}
    \vspace*{\fill} 
    \Huge \textbf{Appendix} \\
    \vspace*{\fill}
\end{center}
\clearpage

\appendix
\addcontentsline{toc}{chapter}{Appendix} 
\fancyhead[]{}
\fancyhead[LO,RE]{\textbf{SHA-256 Hash Function Algorithm}}
\fancyhead[LE,RO]{\thepage}

\renewcommand{\thesection}{\Alph{section}}
\section*{\centering Appendix A: SHA-256 Hash Function Algorithm}
\addcontentsline{toc}{section}{Appendix A: SHA-256 Hash Function Algorithm}

The SHA-256 algorithm is a cryptographic hash function that produces a fixed-size 256-bit hash value from input data. It is used widely for data integrity verification and digital signatures. Below is the description of the SHA-256 algorithm and its mathematical representation.

\begin{algorithm}
\caption{SHA-256 Hash Function}
\begin{algorithmic}
\State \textbf{Input:} A message \( M \) of any length
\State \textbf{Output:} A fixed-length hash value \( H \) (256 bits)
\State Initialize hash values:
\State \( h_0 = \text{0x6a09e667f3bcc908} \)
\State \( h_1 = \text{0xbb67ae85a4a1f7e9} \)
\State \( h_2 = \text{0x3c6ef372fe94f82b} \)
\State \( h_3 = \text{0xa54ff53a5f1d36f1} \)
\State \( h_4 = \text{0x510e527fade682d1} \)
\State \( h_5 = \text{0x9b05688c2b3e6c1f} \)
\State \( h_6 = \text{0x1f83d9abfb41bd6b} \)
\State \( h_7 = \text{0x5be0cd19137e2179} \)
\State \textbf{Padding:} Add padding to the message \( M \) to make its length congruent to 448 mod 512.
\State Append the length of \( M \) (in bits) as a 64-bit integer.
\State \textbf{Processing blocks:} Process the padded message in 512-bit blocks.
\For{each block \( M_i \) in the padded message}
    \State Break \( M_i \) into 16 32-bit words \( W_0, W_1, ..., W_{15} \)
    \For{j = 16 to 63}
        \State \( W_j = \sigma_1(W_{j-2}) + W_{j-7} + \sigma_0(W_{j-15}) + W_{j-16} \)
    \EndFor
    \State Initialize working variables:
    \State \( a = h_0 \), \( b = h_1 \), \( c = h_2 \), \( d = h_3 \), \( e = h_4 \), \( f = h_5 \), \( g = h_6 \), \( h = h_7 \)
    \For{j = 0 to 63}
        \State Compute \( T_1 = h + \Sigma_1(e) + Ch(e,f,g) + K_j + W_j \)
        \State \( T_2 = \Sigma_0(a) + Maj(a,b,c) \)
        \State Update working variables:
        \State \( h = g \), \( g = f \), \( f = e \), \( e = d + T_1 \), \( d = c \), \( c = b \), \( b = a \), \( a = T_1 + T_2 \)
    \EndFor
    \State Update hash values:
    \State \( h_0 = h_0 + a \), \( h_1 = h_1 + b \), \( h_2 = h_2 + c \), \( h_3 = h_3 + d \), 
    \State \( h_4 = h_4 + e \), \( h_5 = h_5 + f \), \( h_6 = h_6 + g \), \( h_7 = h_7 + h \)
\EndFor
\State \textbf{Output:} The final hash value is \( H = h_0 \parallel h_1 \parallel h_2 \parallel h_3 \parallel h_4 \parallel h_5 \parallel h_6 \parallel h_7 \).
\end{algorithmic}
\end{algorithm}

% Start Appendix B
\clearpage
\fancyhead[]{}
\fancyhead[LO,RE]{\textbf{Mathematical Equation Representation of SHA-256}}
\fancyhead[LE,RO]{\thepage}
\section*{\centering Appendix B: Mathematical Equation Representation of SHA-256}
\addcontentsline{toc}{section}{Appendix B: Mathematical Equation Representation of SHA-256}

The SHA-256 function involves several mathematical operations, including bitwise operations, modular additions, and logical functions. The core operations are defined as:

\[
T_1 = h + \Sigma_1(e) + Ch(e,f,g) + K_j + W_j
\]

\[
T_2 = \Sigma_0(a) + Maj(a,b,c)
\]

Where:
- \( \Sigma_1(x) \) and \( \Sigma_0(x) \) are the functions that represent the bitwise rotations and shifts.
- \( Ch(x, y, z) \) is the "choose" function, which selects between values based on certain conditions.
- \( Maj(x, y, z) \) is the "majority" function, which returns the majority bit from the three input bits.

Finally, the 256-bit output hash value is obtained by concatenating the hash values after processing all blocks:

\[
H = h_0 \parallel h_1 \parallel h_2 \parallel h_3 \parallel h_4 \parallel h_5 \parallel h_6 \parallel h_7
\]

% Start Appendix C
\clearpage
\fancyhead[]{}
\fancyhead[LO,RE]{\textbf{Output of the Hash Function}}
\fancyhead[LE,RO]{\thepage}
\section*{\centering Appendix C: Output of the Hash Function}
\addcontentsline{toc}{section}{Appendix C: Output of the Hash Function}

The output of a hash function like SHA-256 is a 256-bit (32-byte) fixed-length value, irrespective of the input size. This means that even if the input data is extremely large, the output will always be a fixed 256-bit hash.

For example, applying SHA-256 to the string "Hello, World!" produces the following hash output:

\[
\text{SHA-256("Hello, World!")} = \text{dffd6021bb2bd7a3643b4d5c891d94b6f97c2bc0c5e9b8482f5a03b4777e0e8f}
\]
\clearpage




\newtheorem{theorem}{Theorem}[section] 
\newtheorem{corollary}[theorem]{Corollary}  

% Start Appendix D
\clearpage
\fancyhead[]{}
\fancyhead[LO,RE]{\textbf{Theorem and Corollary Related to SHA-256}}
\fancyhead[LE,RO]{\thepage}
\section*{\centering Appendix D: Theorem and Corollary Related to SHA-256}
\addcontentsline{toc}{section}{Appendix D: Theorem and Corollary Related to SHA-256}


\setcounter{theorem}{0}
\renewcommand{\thetheorem}{D.\arabic{theorem}}  

% Theorem about SHA-256
\begin{theorem}[Preimage Resistance of SHA-256]
    The SHA-256 hash function is preimage resistant. This means that for any given hash value \( H \), it is computationally infeasible to find any input message \( M \) such that \( \text{SHA-256}(M) = H \).
\end{theorem}

% Corollary related to Preimage Resistance
\begin{corollary}
    If the SHA-256 hash function is preimage resistant, then for a randomly chosen hash output \( H \), finding an input message \( M \) that hashes to \( H \) is equivalent to solving a one-way function, which is computationally infeasible.
\end{corollary}
